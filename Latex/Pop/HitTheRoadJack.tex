\chapter[Hit The Road Jack]{Hit The Road Jack\\[1ex]\large{by Percy Mayfield, but everyone knows the version by Ray Charles}}

\begin{wrapfigure}{R}{0.3\textwidth}
\includegraphics[width=1\linewidth]{QR_Codes/QR_HitTheRoadJack_Chords.png}\\
Chords and Lyrics for C tuned Instruments
\end{wrapfigure}

Lets play it in A minor, so everyone has a great time ;-)\\
\textcolor{lightgray}{In the annals of musical history, "Hit The Road Jack" emerges as an enchanting rhythmic odyssey that takes its listeners on a journey through the sultry depths of jazz and blues. Ray Charles, the maestro behind this timeless composition, weaves a tale of departure and defiance, encapsulating the essence of the human experience with unparalleled vocal prowess and musical finesse.
The very heartbeat of this composition lies in its infectious rhythm, a syncopated dance that mirrors the unpredictable twists and turns of life. From the moment the first notes hit the air, the listener is drawn into the pulsating cadence, echoing the footsteps of a protagonist bidding farewell to a chapter of their life. Charles' gravelly yet soulful voice serves as a guide through this sonic adventure, expressing the raw emotion of a farewell with a playful defiance that is both poignant and irresistible.
} The Progression stays Am, G, F, E7 for the whole Song.
\textcolor{lightgray}{The call-and-response dynamics within "Hit The Road Jack" are reminiscent of a lively conversation between vocalist and instrumentalists, each playing their part in a musical dialogue that mirrors the tumultuous exchanges within a strained relationship. The backing vocalists serve as the chorus of unsolicited advice or perhaps the echoing thoughts of a departing lover, adding a layer of complexity to the narrative. Meanwhile, the instrumentalists, with their nuanced interplay, contribute to the emotional landscape, creating a sonic atmosphere that captures the essence of heartbreak.}\\

Notes in A minor:\\

A, B, C, D, E, F and G (the white keys)
