\chapter[Sixteen Tons]{Sixteen Tons\\[1ex]\large{by Merle Travis}}

\begin{wrapfigure}{R}{0.3\textwidth}
\includegraphics[width=1\linewidth]{QR_Codes/QR_SixteenTons_Chords.png}\\
Chords and Lyrics for C tuned Instruments, The Johnny Cash Version
\end{wrapfigure}

You remember Hit The Road Jack? It's the same, but a bit different\\
\textcolor{lightgray}{"Sixteen Tons," a classic folk song written by Merle Travis and popularized by Tennessee Ernie Ford, stands as a poignant testament to the struggles and hardships faced by the working class, particularly those in the coal mining industry. With its gritty lyrics and soulful delivery, the song encapsulates the gritty reality of a life spent toiling away in the unforgiving depths of the earth.}\\ lets go Am, G, F, E7 (2x)\\ Am, G, Dm, F and Am, Am E, Am for Verse and Chorus.\\ Make a short break after the Chorus and start the next workday.
\textcolor{lightgray}{The title itself, "Sixteen Tons," refers to the immense burden carried by the coal miner—both in terms of the tonnage of coal extracted and the weight of the oppressive working conditions. The opening line, "Some people say a man is made out of mud," sets the tone for the narrative, highlighting the dehumanizing nature of the labor and the toll it takes on the miner's body and spirit.}\\

Notes in A minor:\\

A, B, C, D, E, F and G (the white keys)
